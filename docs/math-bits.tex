% Created 2025-04-14 Mon 11:53
% Intended LaTeX compiler: pdflatex
\documentclass[11pt]{article}
\usepackage[utf8]{inputenc}
\usepackage[T1]{fontenc}
\usepackage{graphicx}
\usepackage{grffile}
\usepackage{longtable}
\usepackage{wrapfig}
\usepackage{rotating}
\usepackage[normalem]{ulem}
\usepackage{amsmath}
\usepackage{textcomp}
\usepackage{amssymb}
\usepackage{capt-of}
\usepackage{hyperref}
\usepackage{pdfrender, amsmath, amsthm, amssymb, calrsfs, wasysym, verbatim, bbm, color, graphics, geometry, fancyhdr, upgreek, mathrsfs, physics}
\usepackage[parfill]{parskip}
\usepackage[inline]{enumitem}
\geometry{tmargin=1in, bmargin=.75in, lmargin= .9in, rmargin = .9in}
\pagestyle{fancy}
\fancyhf{}
\fancyhead[L]{\rightmark}
\rhead{\thepage}
\graphicspath{ {./} }
\author{690 Team}
\date{\today}
\title{Math Bits}
\hypersetup{
 pdfauthor={690 Team},
 pdftitle={Math Bits},
 pdfkeywords={},
 pdfsubject={},
 pdfcreator={Emacs 27.1 (Org mode 9.3)}, 
 pdflang={English}}
\begin{document}

\maketitle

\section*{Costs For Determining Preemption}
\label{sec:org8e6e779}

Our algorithm involves two different costs which are used to evaluate whether or not a job \(j_{i}\) is killed on preemption---cost of killing (\(C_{k}\)) and cost of using a checkpoint with a potential migration to a new processor (\(C_{p}\)). 

Even if we kill a process, it will eventualy need to be re-run. Therefore the cost of killing involves a factor of how much time a job has already been running as whatever work a processor has already done on that job will have to be done again. Because the assumption is that there is no overhead associated with killing, we can say that \(C_{k}\) is simply

\begin{gather*}
C_{k} = t_{i} \\
\end{gather*}

where \(t_{i}\) is the amount of time job \(j_{i}\) has been worked on.

We now consider the cost of using a previous checkpoint. Assume we checkpoint at interval \(l\). If we use a previous checkpoint, then we only have to re-run the job from a previous checkpoint as opposed to the whole job. However we also have to consider a migration cost. In the event a job switches to a new processor, an amount of overhead is associated with writing the memory of the checkpoint to the new processor. If we assume that a processor has an equal likelihood of being scheduled onto any processor, then the probability of incurring a migration cost is \((m - 1)/m\) where \(m\) is the number of processors. Therefore our cost \(C_{p}\) is 

\begin{gather*}
C_{p} = \left( t_{i} - t_{p} \right) + \frac{o_{m} \left( m - 1 \right)}{m}
\end{gather*}

where \(t_{p}\) is the amount of time job \(j_i\) has been worked on at the most recent checkpoint and \(o_{m}\) is the overhead associated with migration. This value is constant. We know that \(t_{i} - t_{p}\) must be bound by \(l\) and therefore we can say that

\begin{gather*}
C_{p} < l + \frac{o_{m} \left( m - 1 \right)}{m}. \\
\end{gather*} 

For later analysis it is important to note that the mean and standard deviations for amount of migrations with \(n_{p}\) preemptions using checkpoints are

\begin{align*}
\mu_{p} &= n_{p} \frac{m - 1}{m} \\
\sigma_{p} &= \frac{\sqrt{n_{p} \left( m - 1 \right)}}{m}. \\
\end{align*}
\end{document}
